\documentclass[12pt]{article}
\usepackage[utf8]{inputenc}
\usepackage[dvips]{graphicx}
\usepackage{epsfig}
\usepackage{fancybox}
\usepackage{verbatim}
\usepackage{array}
\usepackage{latexsym}
\usepackage{alltt}
\usepackage{float}
\usepackage{amsmath}
\usepackage{hyperref}
\usepackage{listings}
\usepackage{color}
\usepackage[hmargin=3cm,vmargin=5.0cm]{geometry}
\topmargin=-1.8cm
\addtolength{\textheight}{6.5cm}
\addtolength{\textwidth}{2.0cm}
\setlength{\oddsidemargin}{0.0cm}
\setlength{\evensidemargin}{0.0cm}

\newcommand{\HRule}{\rule{\linewidth}{1mm}}
\newcommand{\kutu}[2]{\framebox[#1mm]{\rule[-2mm]{0mm}{#2mm}}}
\newcommand{\gap}{ \\[1mm] }

\newcommand{\Q}{\raisebox{1.7pt}{$\scriptstyle\bigcirc$}}

\lstset{
    %backgroundcolor=\color{lbcolor},
    tabsize=2,
    language=C++,
    basicstyle=\footnotesize,
    numberstyle=\footnotesize,
    aboveskip={0.0\baselineskip},
    belowskip={0.0\baselineskip},
    columns=fixed,
    showstringspaces=false,
    breaklines=true,
    prebreak=\raisebox{0ex}[0ex][0ex]{\ensuremath{\hookleftarrow}},
    %frame=single,
    showtabs=false,
    showspaces=false,
    showstringspaces=false,
    identifierstyle=\ttfamily,
    keywordstyle=\color[rgb]{0,0,1},
    commentstyle=\color[rgb]{0.133,0.545,0.133},
    stringstyle=\color[rgb]{0.627,0.126,0.941},
}


\begin{document}



\noindent
\HRule \\[3mm]
\small
\begin{tabular}[b]{lp{3.8cm}r}
{} Middle East Technical University &  &
{} Department of Computer Engineering \\
\end{tabular} \\
\begin{center}

                 \LARGE \textbf{CENG 223} \\[4mm]
                 \Large Discrete Computational Structures \\[4mm]
                \normalsize Fall '2020-2021 \\
                    \Large Homework 3 \\
                \normalsize Student Name and Surname: Yahya SUNGUR  \\
                \normalsize Student Number: 2375723  \\
\end{center}
\HRule


\section*{Question 1}
According to the Fermat's Little Theorem; \\
If p is a prime number and a is an integer not divisible by p, then: \\
$a^{p-1} \equiv 1 (mod$ $p)$ Furthermore, $a^{p} \equiv a (mod$ $p)$ \\
Since,
$$
    2^{22} = (2^{10})^2.2^2 = 1^2.2^2 \equiv 4 (mod 11)
$$
$$
    4^{44} = (4^{10})^4.4^4 = 1^4.4^4 = 256 \equiv 3 (mod 11)
$$
$$
    6^{66} = (6^{10})^6.6^6 = 1^6.6^6 = 46656 \equiv 5 (mod 11)
$$
$$
    8^{80} = (8^{10})^8 = 1^8 \equiv 1 (mod 11)
$$
$$
    10^{110} = (2^{10})^{11} = 1^{11} \equiv 1 (mod 11)
$$
So,
$$
    4+3+5+1+1 = 14 \equiv 3 (mod 11)
$$
Answer = 3

\section*{Question 2}
\begin{itemize}
    \item $7n + 4 = (5n + 3).1 + (2n + 1)$
    \item $5n + 3 = (2n + 1).2 + (n + 1)$
    \item $2n + 1 = (n + 1).1 + n$
    \item $n + 1 = n.1 + 1$
    \item $n = 1.n + 0$
\end{itemize}
By Euclid's Algorithm, \\
gcd(7n+4,5n+3) = gcd(5n+3,2n+1) = gcd(2n+1,n+1) = gcd(n+1,n) = gcd(n,1) = 1 \\
$\implies$ gcd(5n+3,7n+4) = 1

\section*{Question 3}
Since, $m^2 = n^2 + kx$\\
$m^2 - n^2 = kx$ So, both k and x are factors of $(m^2-n^2)$\\
Therefore, both of them divide $(m^2-n^2)$. In other words, \\
$k|(m^2-n^2)$ and $x|(m^2-n^2)$ and since $(m^2-n^2) = (m-n)\times(m+n)$\\
$x|(m^2-n^2) \implies x|(m-n).(m+n)$\\
So, x is a factor of (m-n) or (m+n) \\
Hence $x|(m-n)$ or $x|(m+n)$
\section*{Question 4}
Basis: \\
For n = 1,
$$
 S_1:  1 = \frac{1.(3-1)}{2} \ is\ true.
$$
Inductive step: \\
Assume that;
\[
    S_k: 1 + 4 + 7 + ... + (3k-2) = \frac{k.(3k-1)}{2} \ is\ true.
\]
Prove that,
\[
    S_{k+1}: 1 + 4 + 7 + ... + (3(k+1)-2) = \frac{(k+1).(3(k+1)-1)}{2} \ is\ true.
\]
Observe that,
$$
1 + 4 + 7 + ... + (3(k+1)-2) = 1 + 4 + 7 + ... + (3k-2) + (3k+1)
$$
due to inductive hypothesis,
$$
= \frac{k.(3k-1)}{2} + (3k + 1)\\
$$
$$
= \frac{k.(3k-1)+2.(3k+1)}{2}\\
$$
$$
= \frac{3k^2-k+6k+2)}{2} \\
$$
$$
= \frac{3k^2+3k+2k+2)}{2} \\
$$
$$
= \frac{(k+1).(3k+2)}{2} \\
$$
$$
= \frac{(k+1).(3(k+1)-1)}{2} \\
$$
Thus by the Principle of Mathematical Induction $S_n$ is true for all n $\geq$ 1

\end{document}

