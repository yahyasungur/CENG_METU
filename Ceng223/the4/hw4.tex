\documentclass[11pt]{article}
\usepackage[utf8]{inputenc}
\usepackage{float}
\usepackage{amsmath}
\usepackage{amssymb}

\usepackage[hmargin=3cm,vmargin=6.0cm]{geometry}
%\topmargin=0cm
\topmargin=-2cm
\addtolength{\textheight}{6.5cm}
\addtolength{\textwidth}{2.0cm}
%\setlength{\leftmargin}{-5cm}
\setlength{\oddsidemargin}{0.0cm}
\setlength{\evensidemargin}{0.0cm}

% symbol commands for the curious
\newcommand{\setZp}{\mathbb{Z}^+}
\newcommand{\setR}{\mathbb{R}}
\newcommand{\calT}{\mathcal{T}}

\begin{document}

\section*{Student Information } 
%Write your full name and id number between the colon and newline
%Put one empty space character after colon and before newline
Full Name : YAHYA SUNGUR \\
Id Number : 2375723 \\
.
% Write your answers below the section tags
\section*{Answer 1}
We can choose 1 star from among 10 stars as C(10,1) = $\frac{10!}{(10-1)!\times1!}$ = 10\\
We can choose 2 habitable planets from among 20 habitable planets as C(20,2) = $\frac{20!}{(20-2)!\times2!}$ = 190\\
We can choose 8 non-habitable planets from among 80 non-habitable planets as C(80,8) = $\frac{80!}{(80-8)!\times8!}$ = 28987537150 \\
We need to order these planets that we have selected in the orbit of the star according to their proximity and distance from the star, with the star at the center.\\
.\\
CASE 1: Eight non-habitable planets between two habitable planets \\
"H: Habitable and N: Non-habitable" \\
If we consider planets as group. For this case we have only one group which is (H,N,N,N,N,N,N,N,N,H) \\
The groups can be ordered in 1! different ways for this case since there is only one group.\\
And in the group, we can order H's in 2! and N's in 8! different ways. \\
So, for this case there are $1!\times2!\times8!$ = 80640 different ways for ordering planets. \\
.\\
CASE 2: Seven non-habitable planets between two habitable planets \\
If we consider planets as groups. For this case we have two groups which are (H,N,N,N,N,N,N,N,H),(N) \\
The groups can be ordered in 2! different ways for this case since there are two groups.\\
And in the first group, we can order H's in 2! and N's in 7! different ways. \\
And in the second group, we can order N's in 1! different way since there is only one N in the group. \\
So, if we consider the choices of 8 non-habitable planets, they can be arranged in $2!\times1!\times2!\times7!\times \binom{8}{7}\times \binom{1}{1}$ = 161280 different ways for this case.\\
.\\
CASE 3: Six non-habitable planets between two habitable planets \\
If we consider planets as groups. For this case we have three groups which are (H,N,N,N,N,N,N,H),(N),(N) \\
The groups can be ordered in 3! different ways for this case since there are three groups.\\
And in the first group, we can order H's in 2! and N's in 6! different ways. \\
And in the second and third groups, we can order N's in 1! different way since there is only one N in the groups. \\
So, if we consider the choices of 8 non-habitable planets, they can be arranged in $3!\times1!\times 1!\times 2!\times6!\times \binom{8}{6}\times \binom{2}{2}$ = 241920 different ways for this case.\\
.\\
Hence by the sum rule, there are 10+190+28987537150 = 28987537350 different ways for choosing star and planets, and there are 80640+161280+241920 = 483840 different ways for cases. And by product rule, there are 28987537350 $\times$ 483840 = 14,025,330,071,424,000 ways for forming a galaxy.
\section*{Answer 2}
The associated linear homogeneous equation is:
$$
a_n - 2a_{n-1} - 152a_{n-2} + 362a_{n-3} = 0
$$
and its characteristic equation is:
$$
\lambda^3-2\lambda^2-15\lambda+36=0
\implies (\lambda-3)^3\times(\lambda+4) = 0
\implies \lambda_1 = 3\ \ (multiplicity\ \ 2),\ \lambda_2 =-4 
$$
So, 
$$
a_n^{(h)} = (An+B).(3)^n + C.(-4)^n
$$
Particular solution:\\
Guess: $a_n^{(p)} = x.2^n + y$ and,
$$
(x.2^n + y) -2(x.2^{n-1} + y) -15(x.2^{n-2} + y) + 36(x.2^{n-3} + y) = 2^n
$$
$$
\implies x.2^n + y -x.2^n -2y - \frac{15}{4}x.2^n -15y + \frac{36}{8}x.2^n + 36y = 2^n
$$
$$
\implies \frac{3}{4}x.2^n + 20y = 2^n \implies y=0\ \ and \ \ x=\frac{4}{3}
$$
So,
$$
a_n^{(p)} = \frac{4}{3}.2^n
$$
Hence,
$$
a_n = a_n^{(h)} + a_n^{(p)} =  (An+B).(3)^n + C.(-4)^n + \frac{4}{3}.2^n
$$



\section*{Answer 3}
Let's consider for base cases,\\
For n = 1:\\
The code may consist of 1,3,5,7,9. So, $a_1 = 5$\\
For n = 2:\\
One of the digits must be odd and it can be chosen in $\binom{5}{1}$ ways. Similar procedure valid for even digit. And they can be ordered in 2! ways. So there are $\binom{5}{1} \times \binom{5}{1} \times 2! = 50$ possibilities. Hence $a_2 = 50$\\
For $a_n$: \\
CASE 1:\\
Let's we have valid code with length (n-1). İf we add an even digit to this valid code, we will obtain another valid code with length (n). \\
For this case, number of valid codes with length (n-1) is $a_{n-1}$ and number of even integer in range [0,9] is 5. So, $|CASE1| = a_{n-1} \times 5 $\\
CASE 2:\\
Let's we have invalid code with length (n-1). İf we add an odd digit to this invalid code, we will obtain a valid code with length (n). \\
For this case, number of invalid codes with length (n-1) is equal to (number of codes - number of valid codes),(i.e. universal set - valid codes). So it is ${10}^{n-1}-a_{n-1}$.And odd integer in range [0,9] is 5. So, $|CASE2| = ({10}^{n-1}-a_{n-1}) \times 5 $\\
Since CASE1 $\wedge$ CASE2 = $\emptyset$, by sum rule,
$$
a_n = |CASE1| + |CASE2| = 5\times{10}^{n-1}
$$
\section*{Answer 4}
$$
a_k = 3a_{k-1}-3a_{k-2}+a_{k-3}
$$
$$
\sum_{k=3}^{\infty} a_k.x^k = \sum_{k=3}^{\infty} (3a_{k-1}-3a_{k-2}+a_{k-3}).x^k
$$
$$
= \sum_{k=3}^{\infty} 3a_{k-1}.x^k + \sum_{k=3}^{\infty} (-3)a_{k-2}.x^k +\sum_{k=3}^{\infty} a_{k-3}.x^k
$$
$$
= 3x \sum_{k=3}^{\infty} a_{k-1}.x^{k-1} -3x^2\sum_{k=3}^{\infty} a_{k-2}.x^{k-2} + x^3 \sum_{k=3}^{\infty} a_{k-3}.x^{k-3}
$$
And since,
$$
\sum_{k=3}^{\infty} a_k.x^k = K(x) - a_0 - a_1 - a_2
$$
$$
3x \sum_{k=3}^{\infty} a_{k-1}.x^{k-1} = 3x.(K(x) - a_0 - a_1)
$$
$$
-3x^2 \sum_{k=3}^{\infty} a_{k-2}.x^{k-2} = -3x^2.(K(x) - a_0)
$$
$$
x^3 \sum_{k=3}^{\infty} a_{k-3}.x^{k-3} = x^3.K(x)
$$
We have,
$$
K(x)-10 = 3x.(K(x) - 4) -3x^2.(K(x) - 1) + x^3.K(x)
$$
So,
$$
K(x).(-x^3+3x^2-3x+1) = 3x^2-12x+10 \implies K(x) = \frac{3x^2-12x+10}{(1-x)^3}
$$
By partial fraction,
$$
K(x) = \frac{3}{1-x}+\frac{6}{(1-x)^2} + \frac{1}{(1-x)^3}
$$
From the table "Useful Generating Functions" in the book (page 542):\\
Since,(row 5 in the table)
$$
G(x) = \frac{1}{1-x} \implies a_k = 1
$$
We have,
$$
3\times \frac{1}{1-x} \implies 3\times1 = 3
$$
Since, (row 8 in the table)
$$
G(x) = \frac{1}{(1-x)^2} \implies a_k = k + 1
$$
We have,
$$
6\times \frac{1}{(1-x)^2} \implies 6\times(k+1) = 6k+6
$$
Since, (row 9 in the table)
$$
G(x) = \frac{1}{(1-x)^n} \implies a_k = C(n+k-1,n-1)
$$
We have,
$$
\frac{1}{(1-x)^3} \implies C(3+k-1,3-1) = C(k+2,2) = \frac{k^2+3k+2}{2}
$$
Hence, the sum of all
$$
a_k = \frac{k^2+15k+20}{2}
$$


\section*{Answer 5}
\paragraph{a.}
.\newline
R is reflexive:\\
Suppose (a, b) is an ordered pair in $Z^+ \times Z^+$. [We have to show that (a, b) R (a, b).]\\
We have a + b = a + b. $\forall\ \ a,b\ \ in\ \ Z^+$.\\ Thus, by definition of R, (a, b) R (a, b). So R is reflexive.\\
.\newline
R is symmetric:\\
Suppose (a, b) and (c, d) are two ordered pairs in $Z^+ \times Z^+$ and (a, b) R (c, d). [We have to show that (c, d) R (a, b).] Since (a, b) R (c, d), a + d = b + c. But this implies that b + c = a + d .\\ 
So, by definition of R, (c, d) R (a, b). Thus, R is symmetric.\\
.\newline
R is transitive:\\
Suppose (a, b),(c, d), and (e, f) are elements of $Z^+ \times Z^+$, (a, b) R (c, d), and (c, d) R (e, f). [We have to show that (a, b) R (e, f).] Since (a, b) R (c, d), a + d = b + c, which means a - b = c - d, and since (c, d) R (e, f), c + f = d + e, which means c - d = e - f. Thus a - b = e - f, which means a + f = b + e.\\
So, by definition of R, (a, b) R (e, f). Therefore R is transitive.\\
Since R is reflexive,symmetric and transitive, R is an equivalence relation.
\paragraph{b.}
$$
[(1,2)] = \{(a,b) \in Z^+ \times Z^+ | (a,b) R (1,2)\}
$$
$$
[(1,2)] = \{(a,b) \in Z^+ \times Z^+ | a+2 = b+1\}
$$
$$
[(1,2)] = \{(a,b) \in Z^+ \times Z^+ | a+1 = b\}
$$
$$
[(1,2)] = \{(a,a+1) | a \in Z^+\}
$$

\end{document}