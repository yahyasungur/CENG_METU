\documentclass[a4paper,12pt]{article}

\usepackage{float}


\usepackage[utf8]{inputenc}
\usepackage[dvips]{graphicx}
%\usepackage{a4wide}
\usepackage{epsfig}
\usepackage{fancybox}
\usepackage{verbatim}
\usepackage{array}
\usepackage{latexsym}
\usepackage{alltt}
\usepackage{amssymb}
\usepackage{amsmath,amsthm}
\usepackage{bm}
\usepackage{wasysym}

%\usepackage{fullpage}
%\usepackage{hyperref}
\usepackage{listings}
\usepackage{color}
\usepackage{algorithm}
\usepackage{algpseudocode}
\usepackage[hmargin=2cm,vmargin=3.0cm]{geometry}
%\topmargin=0cm
%\topmargin=-1.8cm
%\addtolength{\textheight}{6.5cm}
%\addtolength{\textwidth}{2.0cm}
%\setlength{\leftmargin}{-3cm}
%\setlength{\oddsidemargin}{0.0cm}
%\setlength{\evensidemargin}{0.0cm}

%misc libraries goes here
\usepackage{tikz}
\usepackage{tikz-qtree}
\usetikzlibrary{automata,positioning}

\usepackage{multicol}
\usepackage{enumitem}

\usepackage[most]{tcolorbox}

\usepackage[colorlinks=true,urlcolor=black,linkcolor=black]{hyperref}


\lstdefinestyle{customtex}{
    %backgroundcolor=\color{lbcolor},
    tabsize=2,
    language=TeX,
    numbers=none,
    basicstyle=\footnotesize\ttfamily,
    numberstyle=\footnotesize,
    aboveskip={0.0\baselineskip},
    belowskip={0.0\baselineskip},
    %
    columns=flexible,
    keepspaces=true,
    fontadjust=true,
    upquote=true,
    %
    breaklines=true,
    prebreak=\raisebox{0ex}[0ex][0ex]{\ensuremath{\hookleftarrow}},
    frame=single,
    showtabs=false,
    showspaces=false,
    showstringspaces=false,
    %
    %identifierstyle=\color[rgb]{0,0.2,0.8},
    identifierstyle=\color[rgb]{0,0,0.5},
    %identifierstyle=\color[rgb]{0.133,0.545,0.133},
    %keywordstyle=\color[rgb]{0.8,0,0},
    %keywordstyle=\color[rgb]{0.133,0.545,0.133},
    keywordstyle=\color[rgb]{0,0,0.5},
    %commentstyle=\color[rgb]{0.133,0.545,0.133},
    commentstyle=\color[rgb]{0.545,0.545,0.545},
    %stringstyle=\color[rgb]{0.827,0.627,0.133},
    stringstyle=\color[rgb]{0.133,0.545,0.133},
    %
    literate={â}{{\^{a}}}1 {Â}{{\^{A}}}1 {ç}{{\c{c}}}1 {Ç}{{\c{C}}}1 {ğ}{{\u{g}}}1 {Ğ}{{\u{G}}}1 {ı}{{\i}}1 {İ}{{\.{I}}}1   {ö}{{\"o}}1 {Ö}{{\"O}}1 {ş}{{\c{s}}}1 {Ş}{{\c{S}}}1 {ü}{{\"u}}1 {Ü}{{\"U}}1 {~}{$\sim$}{1}
}

\lstdefinestyle{output}{
    %backgroundcolor=\color{lbcolor},
    tabsize=2,
    numbers=none,
    basicstyle=\footnotesize\ttfamily,
    numberstyle=\footnotesize,
    aboveskip={0.0\baselineskip},
    belowskip={0.0\baselineskip},
    %
    columns=flexible,
    keepspaces=true,
    fontadjust=true,
    upquote=true,
    %
    breaklines=true,
    prebreak=\raisebox{0ex}[0ex][0ex]{\ensuremath{\hookleftarrow}},
    frame=single,
    showtabs=false,
    showspaces=false,
    showstringspaces=false,
    %
    %identifierstyle=\color[rgb]{0.44,0.12,0.1},
    identifierstyle=\color[rgb]{0,0,0},
    keywordstyle=\color[rgb]{0,0,0},
    commentstyle=\color[rgb]{0,0,0},
    stringstyle=\color[rgb]{0,0,0},
    %
    literate={â}{{\^{a}}}1 {Â}{{\^{A}}}1 {ç}{{\c{c}}}1 {Ç}{{\c{C}}}1 {ğ}{{\u{g}}}1 {Ğ}{{\u{G}}}1 {ı}{{\i}}1 {İ}{{\.{I}}}1   {ö}{{\"o}}1 {Ö}{{\"O}}1 {ş}{{\c{s}}}1 {Ş}{{\c{S}}}1 {ü}{{\"u}}1 {Ü}{{\"U}}1
}

\lstset{style=customtex}


\tikzset{%
    terminal/.style={draw, rectangle,
    				 align=center, 
					 minimum height=1cm, 
					 minimum width=2cm,
					 fill=black!10,
					 anchor=mid},
    nonterminal/.style={draw, rectangle,
    					align=left,
					    minimum height=1cm, 
						minimum width=2cm, 
						anchor=mid},% and so on
}

%% Style for terminals
%\tikzstyle{terminal}=[draw, rectangle, 
%					  minimum height=1cm, 
%					  minimum width=2cm, 
%					  fill=black!20,
%					  anchor=south west]
%% Style for nonterminals
%\tikzstyle{nonterminal}=[draw, rectangle, 
%						 minimum height=1 cm, 
%						 minimum width=2 cm, 
%						 anchor=north east]


\newcommand{\HRule}{\rule{\linewidth}{1mm}}
\newcommand{\kutu}[2]{\framebox[#1mm]{\rule[-2mm]{0mm}{#2mm}}}
\newcommand{\gap}{ \\[1mm] }

\newcommand{\Q}{\raisebox{1.7pt}{$\scriptstyle\bigcirc$}}
\newcommand{\minus}{\scalebox{0.35}[1.0]{$-$}}

\setlength{\fboxsep}{10pt}

\tcbsetforeverylayer{enhanced jigsaw, breakable, arc=0mm, boxrule=1pt, boxsep=5pt, after=\vspace{1em}, colback=white, colframe=black}

\newcolumntype{P}[1]{>{\centering\arraybackslash}p{#1}}

\setlength\parindent{0pt}
\newcommand\tab[1][1cm]{\hspace*{#1}}

%\renewcommand\arraystretch{1.2}

\newenvironment{Tab}[1]
  {\def\arraystretch{1}\tabular{#1}}
  {\endtabular}

%%%%%%%%%%%%%%%%%%%%%%%%%%%%%%%%%%%%%%%%%%%%%%%%%%%%%%%%%%%%%%%%%%%%%%%%%%%%%%%%%%%%%%

\title{Discrete Computational Structures \\ Take Home Exam 1}
\author{Name SURNAME \\ 1234567} % write your name and id
\date{} % do not write any date

%%%%%%%%%%%%%%%%%%%%%%%%%%%%%%%%%%%%%%%%%%%%%%%%%%%%%%%%%%%%%%%%%%%%%%%%%%%%%%%%%%%%%%

\begin{document}
\HRule\\
Middle East Technical University \hfill Department of Computer Engineering
{\let\newpage\relax\maketitle}
\HRule\\
\vspace{1cm}

%%%%%%%%%%%%%%%%%%%%%%%%%%%%%%%%%%%%%%%%%%%%%%%%%%%%%%%%%%%%%%%%%%%%%%%%%%%%%%%%%%%%%%

% Write your answers below the section tags
\section*{Question 1 \hfill \normalfont{(7 pts)}}

\tab \textbf{a)} Construct a truth table for the following compound proposition.
\begin{equation*}
    (q \rightarrow \neg p)\leftrightarrow (p \leftrightarrow \neg q)
\end{equation*} 
\hfill \small{(3.5/7 pts)} \\
\begin{tcolorbox}
    \begin{center}
        \begin{tabular}{|c|c|c|c|c|c|c|}
            \hline
            p & q & $\neg$ p & $\neg$ q & q $\rightarrow$ $\neg$ p & p $\leftrightarrow$ $\neg$ q & (q $\rightarrow$ $\neg$ p)$\leftrightarrow$ (p $\leftrightarrow$ $\neg$ q)\\
            \hline
            T & T & F & F & F & F & T \\
            T & F & F & T & T & T & T \\
            F & T & T & F & T & T & T \\
            F & F & T & T & T & F & F \\
            \hline
        \end{tabular}
    \end{center}
\end{tcolorbox}
\newpage
\tab \textbf{b)} Show that whether the following conditional statement is a tautology by using a truth table.
\begin{equation*}
    [(p \vee q) \wedge (r \rightarrow p) \wedge (r \rightarrow q)] \rightarrow r
\end{equation*}
\hfill \small{(3.5/7 pts)}\\
\begin{tcolorbox}
    \begin{center}
        \begin{tabular}{|c|c|c|c|c|c|c|}
            \hline
            p & q & r & p $\vee$ q & r $\rightarrow$ p & r $\rightarrow$ q & [(p $\vee$ q) $\wedge$ (r $\rightarrow$ p) $\wedge$ (r $\rightarrow$ q)] $\rightarrow$ r \\
            \hline
            T & T & T & T & T & T & T \\
            T & T & F & T & T & T & F \\
            T & F & T & T & T & F & T \\
            T & F & F & T & T & T & F \\
            F & T & T & T & F & T & T \\
            F & T & F & T & T & T & F \\
            F & F & T & F & F & F & T \\
            F & F & F & F & T & T & T \\
            \hline
        \end{tabular}
    \end{center}
\end{tcolorbox}


\section*{Question 2 \hfill \normalfont{(8 pts)}}
\tab Show that $(p \rightarrow q) \wedge (p \rightarrow r)$ and $(\neg q \vee \neg r)\rightarrow \neg p$ are logically equivalent. Use tables 6,7 and 8 given under the section "$\textit{Propositional Equivalences}$" in the course textbook and give the reference to the table and the law in each step.

\begin{tcolorbox}
\text{}
$(p \rightarrow q) \wedge (p \rightarrow r)$ $\equiv$ $p \rightarrow (q \wedge r)$ \tab\tab (\text{table 7 , and rule $(p \rightarrow q) \wedge (p \rightarrow r)$ $\equiv$ $p \rightarrow (q \wedge r)$ )}
\newline
\tab\tab\tab $\equiv$ $\neg$ $(q \wedge r) \rightarrow \neg p$ \tab (\text{table 7 , and rule $p \rightarrow q$ $\equiv$ $\neg q \rightarrow \neg q$})
\newline
\tab\tab\tab $\equiv$  $(\neg q \vee \neg r) \rightarrow \neg p$ \tab (\text{table 6 , and De Morgan’s laws})
    
    

\end{tcolorbox}

\section*{Question 3 \hfill \normalfont{(30 pts, 2.5 pts each)}}
\tab Let F(x, y) mean that x is the father of y; M(x, y) denotes x is the mother of y. Similarly,
H(x, y), S(x, y), and B(x, y) say that x is the husband/sister/brother of y, respectively. You
may also use constants to denote individuals, like Sam and Alex. You can use $\vee, \wedge, \rightarrow, \neg, \forall, \exists$ rules and quantifiers. However, you are not allowed
to use any predicate symbols other than the above to translate the following sentences into
predicate logic. $\exists$! and exclusive-or (XOR) quantifiers are forbidden:
\begin{multicols}{2}
\begin{enumerate}[label=\textbf{\arabic*})]
\item Everybody has a mother.
\item Everybody has a father and a mother.
\item Whoever has a mother has a father.
\item Sam is a grandfather.
\item All fathers are parents.
\item All husbands are spouses.
\item No uncle is an aunt.
\item All brothers are siblings.
\item Nobody's grandmother is anybody's father.
\item Alex is Ali's brother-in-law.
\item Alex has at least two children.
\item Everybody has at most one mother.
\end{enumerate}
\end{multicols}
\newpage
\begin{tcolorbox}

\begin{enumerate}
    \item $\forall y \exists x$ M(x,y) 
    \item $\forall y \exists x$ (F(x,y) $\wedge$ M(x,y))
    \item $\forall y$ [[$\exists x$ M(x,y)] $\rightarrow [ \exists x$  F(x,y)]]
    \item $\exists x \exists$y (F(Sam,x) $\wedge$ F(x,y))
    \item $\forall x \exists y$ F(x,y) $\rightarrow$ [F(x,y) $\vee$ M(x,y)] 
    \item $\forall x \exists y \exists z$ H(x,y) $\rightarrow$ [H(x,y) $\vee$ H(z,x)]
    \item $\forall x \forall y \exists k$  [B(x,y) $\wedge (F(y,k) \vee M(y,k))]$ $\rightarrow \neg [S(x,y) \wedge (F(y,k) \vee M(y,k))]$
    \item $\forall x \forall y \exists z \exists k$ B(x,y) $\rightarrow [(M(z,x) \wedge M(z,y)) \vee (F(k,x) \wedge F(k,y))] $
    \item $\forall x \forall y \forall z \forall k$ (M(x,y) $\wedge$ F(y,z)) $\rightarrow \neg F(x,k)$
    \item $\forall x \forall y$ [H(Alex,x) $\wedge$ B(Ali,x)]
    \item $\exists x \exists y$ F(Alex,x) $\wedge$ F(Alex,y)
    \item $\forall x \forall y \forall z$ [M(y,x) $\wedge$ M(z,x)] $\rightarrow$ y = z
\end{enumerate}
\end{tcolorbox}
\newpage
\section*{Question 4 \hfill \normalfont{(25 pts)}}
\tab Prove the following claim by natural deduction. Use only the natural deduction rules $\vee$, $\wedge$, $\rightarrow$, $\neg$ introduction and elimination. If you attempt to make use of a lemma or equivalence, you need to prove it by natural deduction too.
\begin{equation*}
    \textbf{a)} \; p \rightarrow q, r \rightarrow s \vdash (p \vee r) \rightarrow (q \vee s)
\end{equation*}
\hfill \small{(12.5/25 pts)}\\
\begin{tcolorbox}
\begin{enumerate}
    \item p $\rightarrow$ q \tab (premise)
    \item r $\rightarrow$ s \tab (premise)
    \begin{tcolorbox}
        \item p $\vee$ r \tab (assumption)
        \begin{tcolorbox}
            \item p \tab (assumption)
            \item q \tab (1,4 and $\rightarrow$ elimination)
            \item q $\vee$ s \tab (5 and $\vee$ introduction)
        \end{tcolorbox}
        \begin{tcolorbox}
            \item r \tab (assumption)
            \item s \tab (2,7 and $\rightarrow$ elimination)
            \item q $\vee$ s \tab (8 and $\vee$ introduction)
        \end{tcolorbox}
        \item q $\vee$ s \tab (3,4-6, 7-9 and $\vee$ elimination)
    \end{tcolorbox}
    \item (p $\vee$ r) $\rightarrow$ (q $\vee$ s) \tab (3-10, $\rightarrow$ introduction)
\end{enumerate}
\end{tcolorbox}
\newpage
\begin{equation*}
     \textbf{b)} \;  (p \rightarrow (r \rightarrow \neg q)) \rightarrow ((p \wedge q) \rightarrow \neg r)
\end{equation*}

\hfill \small{(12.5/25 pts)}\\
\begin{tcolorbox}
\begin{enumerate}
    \item $(p \rightarrow (r \rightarrow \neg q)) \rightarrow ((p \wedge q) \rightarrow \neg r)$
    \item $( \neg (r \rightarrow \neg q) \rightarrow \neg p ) \rightarrow ((p \wedge q) \rightarrow \neg r)$ \tab (contrapositive)
    \item $( \neg (\neg r \vee \neg q) \rightarrow \neg p ) \rightarrow ((p \wedge q) \rightarrow \neg r)$ \tab (equivalence(*) $p \rightarrow q \equiv \neg p \vee q)$
    \item $((r \wedge q) \rightarrow \neg p ) \rightarrow ((p \wedge q) \rightarrow \neg r)$\tab\tab(De Morgan's Law)
    \begin{tcolorbox}
        \item $(r \wedge q) \rightarrow \neg p )$ \tab (assumption)
        \begin{tcolorbox}
            \item $(p \wedge q)$ \tab (assumption)
            \begin{tcolorbox}
                \item r \tab (assumption)
                \item q \tab (6 and $\wedge$ elimination)
                \item r $\wedge$ q \tab (7,8 and $\wedge$ introduction)
                \item $\neg$ p \tab (5,9 and $\rightarrow$ elimination)
                \item p \tab (6 and $\wedge$ elimination)
            \end{tcolorbox}
            \item $\neg$ r \tab (7-11 and falsity elimination (PbC))
        \end{tcolorbox}
        \item (p $\wedge$ q) $\rightarrow \neg$ r \tab (6-12 and $\rightarrow$ introduction)
    \end{tcolorbox}
    \item $((r \wedge q) \rightarrow \neg p ) \rightarrow ((p \wedge q) \rightarrow \neg r)$ \tab (5-13 and $\rightarrow$ introduction)
    \item $(p \rightarrow (r \rightarrow \neg q)) \rightarrow ((p \wedge q) \rightarrow \neg r)$ \tab (1,2,3,4,14)
\end{enumerate}

\begin{center}
--------------------------------------------------------------------------------------------\newline
------- equivalence(*) -------
\end{center}
\begin{enumerate}
    \item $p \rightarrow q$ \tab (premise)
    \begin{tcolorbox}
        \item p \tab (assumption)
        \item q \tab (1,2 and $\rightarrow$ elimination)
        \item $\neg$p $\vee$ q \tab (3 and $\vee$ introduction)
    \end{tcolorbox}
    \begin{tcolorbox}
        \item $\neg$p \tab (assumption)
        \item $\neg$ p $\vee$ q \tab (5 and $\vee$ introduction)
    \end{tcolorbox}
    \begin{tcolorbox}
        \item q \tab (assumption)
        \item $\neg$ p $\vee$ q \tab (7 and $\vee$ introduction)
    \end{tcolorbox}
    \item $\neg$ p $\vee$ q \tab (1,2-4,5-6,7-8,$\rightarrow$ elimination)
\end{enumerate}


\end{tcolorbox}

\section*{Question 5 \hfill \normalfont{(30 pts)}}
\tab Prove the following claim by natural deduction. Use only the natural deduction rules $\vee$, $\wedge$, $\rightarrow$, $\neg$ introduction and elimination. If you attempt to make use of a lemma or equivalence, you need to prove it by natural deduction too.


\begin{equation*}
    \textbf{a)} \; \forall x P(x) \vee \forall x Q(x) \vdash \forall x (P(x) \vee Q(x)) 
\end{equation*}

\hfill \small{(12.5/25 pts)}\\
\begin{tcolorbox}

\begin{enumerate}
    \item $\forall x$ P(x) $\vee$ $\forall x$ Q(x) \tab (premise)
    \begin{tcolorbox}
        \item $\forall x$ P(x) \tab (assumption)
        \begin{tcolorbox}
            a is fresh name
            \item P(a) \tab (2 and $\forall$ elimination)
            \item P(a) $\vee$ Q(a) \tab (3 and $\vee$ introduction)
        \end{tcolorbox}
        \item $\forall x$ (P(x) $\vee$ Q(x)) \tab   (3-4 and $\forall$ introduction)
    \end{tcolorbox}
    \begin{tcolorbox}
        \item $\forall x$ Q(x) \tab (assumption)
        \begin{tcolorbox}
            b is fresh name
            \item Q(b) \tab (6 and $\forall$ elimination)
            \item P(b) $\vee$ Q(b) \tab (7 and $\vee$ introduction)
        \end{tcolorbox}
        \item $\forall x$ (P(x) $\vee$ Q(x)) \tab   (7-8 and $\forall$ introduction)
    \end{tcolorbox}
    \item $\forall x$ (P(x) $\vee$ Q(x)) \tab   (2-5,6-9 and $\vee$ elimination)
\end{enumerate}

\end{tcolorbox}
\newpage
\begin{equation*}
    \textbf{b)} \; \forall x P(x) \rightarrow S \vdash \exists x (P(x)\rightarrow S)\\
\end{equation*}

\hfill \small{(17.5/25 pts)}\\
\begin{tcolorbox}
\begin{enumerate}
    \item $\forall x$ P(x) $\rightarrow$ S  \tab (premise)
    \item $\neg$($\forall x$ P(x)) $\vee$ S  \tab (1 and equivalence(*) $A \rightarrow B$ $\equiv$ $\neg A \vee B$)
    \item ($\exists x$ $\neg$P(x)) $\vee$ S  \tab (2 and equivalance(**) $ \neg \forall x A(x) \equiv \exists x \neg A(x)$)
    \begin{tcolorbox}
        \item $\exists x$ $\neg$P(x) \tab (assumption)
        \begin{tcolorbox}
            a is fresh name
            \item $\neg$P(a) \tab (4 and $\exists$ elimination)
            \item $\neg$P(a) $\vee$ S \tab (5 and $\vee$ introduction)
            \item P(a) $\rightarrow$ S \tab (6 and equivalence(*) $A \rightarrow B$ $\equiv$ $\neg A \vee B$)
        \end{tcolorbox}
        \item $\exists x (P(x)\rightarrow S)$ \tab   (5-7 and $\exists$ introduction)
    \end{tcolorbox}
    \begin{tcolorbox}
        \item S \tab (assumption)
        \begin{tcolorbox}
            b is fresh name
            \item $\neg$P(b) $\vee$ S \tab (9 and $\vee$ introduction)
            \item P(b) $\rightarrow$ S \tab (10 and equivalence(*) $A \rightarrow B$ $\equiv$ $\neg A \vee B$)
        \end{tcolorbox}
        \item $\exists x (P(x)\rightarrow S)$ \tab   (10-11 and $\exists$ introduction)
    \end{tcolorbox}
    \item $\exists x (P(x)\rightarrow S)$ \tab   (3,4-8,9-12 and $\vee$ elimination)
\end{enumerate}
\begin{center}
--------------------------------------------------------------------------------------------\newline
------- equivalence(*) -------
\end{center}
\begin{enumerate}
    \item $p \rightarrow q$ \tab (premise)
    \begin{tcolorbox}
        \item p \tab (assumption)
        \item q \tab (1,2 and $\rightarrow$ elimination)
        \item $\neg$p $\vee$ q \tab (3 and $\vee$ introduction)
    \end{tcolorbox}
    \begin{tcolorbox}
        \item $\neg$p \tab (assumption)
        \item $\neg$ p $\vee$ q \tab (5 and $\vee$ introduction)
    \end{tcolorbox}
    \begin{tcolorbox}
        \item q \tab (assumption)
        \item $\neg$ p $\vee$ q \tab (7 and $\vee$ introduction)
    \end{tcolorbox}
    \item $\neg$ p $\vee$ q \tab (1,2-4,5-6,7-8,$\rightarrow$ elimination)
\end{enumerate}
\begin{center}
--------------------------------------------------------------------------------------------\newline
------- equivalence(**)----[left to right] -------
\end{center}

\begin{enumerate}
    \item $\neg \forall x P(x)$ \tab (premise)
    \begin{tcolorbox}
        a fresh name
        \item P(a) \tab (assumption)
        \item $\forall x$P(x) \tab (2 and $\forall$ introduction)
        \item F \tab (1,3)
    \end{tcolorbox}
    \begin{tcolorbox}
        b fresh name
        \item $\neg$P(b) \tab (2-4 and $\neg$ introduction)
    \end{tcolorbox}
    \item $\exists x \neg$P(x) \tab (5 and $\exists$ introduction)
\end{enumerate}
\begin{center}
------- equivalence(**)----[right to left] -------
\end{center}
\begin{enumerate}
    \item $\exists x \neg$P(x) \tab (premise)
    \begin{tcolorbox}
        \item $\forall x$P(x) \tab (assumption)
        \begin{tcolorbox}
        a fresh name
            \item $\neg$P(a) \tab (1 and $\exists$ elimination)
            \item P(a) \tab (2 and $\forall$ elimination)
            \item F \tab (3,4)
        \end{tcolorbox}
        \item F \tab (2, 3-5)
    \end{tcolorbox}
    \item $\neg \forall x P(x)$ \tab (1,2-6, $\neg$ introduction)
\end{enumerate}

\end{tcolorbox}

\end{document}
